\documentclass[]{article}
\usepackage{lmodern}
\usepackage{amssymb,amsmath}
\usepackage{ifxetex,ifluatex}
\usepackage{fixltx2e} % provides \textsubscript
\ifnum 0\ifxetex 1\fi\ifluatex 1\fi=0 % if pdftex
  \usepackage[T1]{fontenc}
  \usepackage[utf8]{inputenc}
\else % if luatex or xelatex
  \ifxetex
    \usepackage{mathspec}
  \else
    \usepackage{fontspec}
  \fi
  \defaultfontfeatures{Ligatures=TeX,Scale=MatchLowercase}
\fi
% use upquote if available, for straight quotes in verbatim environments
\IfFileExists{upquote.sty}{\usepackage{upquote}}{}
% use microtype if available
\IfFileExists{microtype.sty}{%
\usepackage{microtype}
\UseMicrotypeSet[protrusion]{basicmath} % disable protrusion for tt fonts
}{}
\usepackage[margin=1in]{geometry}
\usepackage{hyperref}
\hypersetup{unicode=true,
            pdftitle={Decision Theory Assignment 1},
            pdfauthor={Omkar Bhutra (omkbh878)},
            pdfborder={0 0 0},
            breaklinks=true}
\urlstyle{same}  % don't use monospace font for urls
\usepackage{graphicx,grffile}
\makeatletter
\def\maxwidth{\ifdim\Gin@nat@width>\linewidth\linewidth\else\Gin@nat@width\fi}
\def\maxheight{\ifdim\Gin@nat@height>\textheight\textheight\else\Gin@nat@height\fi}
\makeatother
% Scale images if necessary, so that they will not overflow the page
% margins by default, and it is still possible to overwrite the defaults
% using explicit options in \includegraphics[width, height, ...]{}
\setkeys{Gin}{width=\maxwidth,height=\maxheight,keepaspectratio}
\IfFileExists{parskip.sty}{%
\usepackage{parskip}
}{% else
\setlength{\parindent}{0pt}
\setlength{\parskip}{6pt plus 2pt minus 1pt}
}
\setlength{\emergencystretch}{3em}  % prevent overfull lines
\providecommand{\tightlist}{%
  \setlength{\itemsep}{0pt}\setlength{\parskip}{0pt}}
\setcounter{secnumdepth}{0}
% Redefines (sub)paragraphs to behave more like sections
\ifx\paragraph\undefined\else
\let\oldparagraph\paragraph
\renewcommand{\paragraph}[1]{\oldparagraph{#1}\mbox{}}
\fi
\ifx\subparagraph\undefined\else
\let\oldsubparagraph\subparagraph
\renewcommand{\subparagraph}[1]{\oldsubparagraph{#1}\mbox{}}
\fi

%%% Use protect on footnotes to avoid problems with footnotes in titles
\let\rmarkdownfootnote\footnote%
\def\footnote{\protect\rmarkdownfootnote}

%%% Change title format to be more compact
\usepackage{titling}

% Create subtitle command for use in maketitle
\providecommand{\subtitle}[1]{
  \posttitle{
    \begin{center}\large#1\end{center}
    }
}

\setlength{\droptitle}{-2em}

  \title{Decision Theory Assignment 1}
    \pretitle{\vspace{\droptitle}\centering\huge}
  \posttitle{\par}
    \author{Omkar Bhutra (omkbh878)}
    \preauthor{\centering\large\emph}
  \postauthor{\par}
      \predate{\centering\large\emph}
  \postdate{\par}
    \date{2 November 2019}


\begin{document}
\maketitle

\section{Assignment 1:}\label{assignment-1}

\subsection{1}\label{section}

\subsubsection{a}\label{a}

Using Likihood as a measure of how likely is an event is a matter of
inference to the best explanation. Inference to the best explanation:
\(P(B|\overline{A},I) = 0.1%\) \(P(B|A_1,I) = 2%\) \(P(B|A_2,I) = 10%\)
\(P(B|A_3,I) = 20%\)

\(P(B|A_1,I) > P(B|A_j,I)\) Then \(A_1\) is considered bset explanation
for B and is provisionally accepted Hence, GP's diagnosis using the
principles of inference to the best explanation is that disease \(A_1\)
explains the Symptoms.

\subsection{1}\label{section-1}

\subsubsection{b}\label{b}

\paragraph{i}\label{i}

\(P(A_1) = 0.2%\) \(P(\overline A_1) = 99.8%\) \(P(A_2) = 0.3%\)
\(P(\overline A_2) = 99.7%\) \(P(A_3) = 1%\) \(P(\overline A_3) = 99%\)
\(P(B|A_1,I) = 2%\) \(P(B|\overline A_1,I) = 98%\) \(P(B|A_2,I) = 10%\)
\(P(B|\overline A_2,I) = 90%\) \(P(B|A_3,I) = 20%\)
\(P(B|\overline A_3,I) = 80%\)

Conditional probabilities:
\(P(A|B) = \frac{P(B|A).P(A)}{P(B|A).P(A)+P(B|\overline A).P(\overline A)}\)
(i) \$ \frac{0.02X0.002}{0.02X0.002+0.98X0.998} = 97.8\%\$ (ii) \$
\frac{0.1X0.003}{0.1X0.003+0.9X0.997} = 89.1\%\$ (iii) \$
\frac{0.2X0.01}{0.2X0.01+0.8X0.99} = 79.2\%\$

\paragraph{ii}\label{ii}

Since, \(P(A_1|B) > P(A_j|B)\) Then \(A_1\) is considered bset
explanation for the symptopms and is provisionally accepted.

\subsection{2}\label{section-2}

\subsubsection{i}\label{i-1}

Posterior probability model for Binomial sampling: The sample size n
could be fixed, in which case \(\tilde r\) is a random variable and the
distribution of \(\tilde r\) given p and n is a binomial distribution.

\(P(p| n, r) \propto \left[\begin{array}{cc} n \\r \end{array}\right] p^r(1-p)^{n-r} P(p)\)

For Pascal sampling, the number of successes r could be fixed (sample
until r successes are obtained), in which case \(\tilde n\) is a random
variable and the distribution of \(\tilde n\) given p and r is a Pascal
distribution, which is of the following form: \[
P(p| n, r) \propto \left(\begin{array}{cc} 
n-1 \\
r-1
\end{array}\right)
p^r(1-p)^{n-r}.P(p)
\]

These two distributions differ only in the first terms, the
combinatorial terms. The reason for this is that the last trial must be
a success in Pascal sampling. In the posterior, The combinatorial terms
are irrelevant since they do not involve \(\tilde p\). The combinatorial
term can be moved outside the integral where it appears in both the
numerator and denominator and can be cancelled out. Therefore, the
liklihood function fot the Bernoulli process can be taken equal to
\(p^r(1-p)^{n-r}\) so that it isnt necessary to know whether the
sampling is done with n fixed or with r fixed.

Consider the equation:
\(f(\theta|y) = \frac{f(y|\theta)f(\theta)}{\left\int_{-\infty}^{\infty} f(y|\theta)f(\theta) \; d\theta\right}\)

\$f''(p\textbar{}y) = \frac{f'(p)f(y|p)}{\int_{0}^{1}f'(p)f(y|p)dp} =
\frac{f'(p)P(r|n,p)}{\int_{0}^{1}f'(p)P(r|n,p)dp} \$

\[= \frac{\frac{(n'-1)!}{(r'-1)!(n'-r'-1)!}p^{r'-1}(1-p)^{n'-r'-1}\left(\begin{array}{cc} 
n \\
r 
\end{array}\right)p^r(1-p^{n-r})}
{\int_{0}^{1}\frac{(n'-1)!}{(r'-1)!(n'-r'-1)!}p^{r'-1}(1-p)^{n'-r'-1}\left(\begin{array}{cc} 
n \\
r 
\end{array}\right)p^r(1-p^{n-r})dp}\]

The terms not involving p can be moved outside the integral

\$f''(p\textbar{}y) =
\frac{p^{r'-1}(1-p)^{n'-r'-1}p^r(1-p)^{n-r}}{\int_{0}^{1}p^{r'-1}(1-p)^{n'-r'-1}p^r(1-p)^{n-r}dp}
\$

\$f'`(p\textbar{}y) =
\frac{1}{k}p\textsuperscript{\{r'+r-1\}(1-p)}\{n'+n-r'-r-1\} \$

\(n''=n'+n ; r''=r'+r\)

\(f''(p|y) = \frac{1}{k}p^{r''-1}(1-p)^{n''-r''-1}\)

\(k = \frac{(r''-1)!(n''-r''-1)!}{(n''-1)!}\) Either case, the term
cancels out so the final result is the same

\subsubsection{ii}\label{ii-1}

To determine the posterior distribution, the only information needed
about the sample consists of the values of the sample statistics, r and
n. Hence, r and n are sufficient statistics. The procedure is used to
tell statisticians when to stop sampling is called the stopping rule, if
the stopping rule has no effect on the posterior distribution, then it
is said to be non-informative. The two stopping rules are: (sample until
n trails and sample until r successes ) are both non-informative.


\end{document}
